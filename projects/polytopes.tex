\documentclass{article}


\input{../../template/tex/general-tex/usepackage.tex} 
\addbibresource{sample.bib} %Import the bibliography file

\newcommand{\commentt}[1]{\textcolor{blue}{ \textbf{[COMMENT]} #1}}
\newcommand{\ctt}[1]{\commentt{#1}}
\newcommand{\prb}[1]{ \mathbf{Pr} \left[ {#1} \right]}
\newcommand{\expp}[1]{ \mathbf{E} \left[ {#1} \right]}
\newcommand{\onotation}[1]{\(\mathcal{O} \left( {#1}  \right) \)}
\newcommand{\ona}[1]{\onotation{#1}}
\newcommand{\PSI}{{\ket{\psi}}}
\newcommand{\LESn}{\ket{\psi_n}}
\newcommand{\LESa}{\ket{\phi_n}}
\newcommand{\LESs}{\frac{1}{\sqrt{n}}\sum_{i}{\ket{\left(0^{i}10^{n-i}\right)^{n}}}}
\newcommand{\Hn}{\mathcal{H}_{n}}
\newcommand{\Ep}{\frac{1}{\sqrt{2^n}}\sum^{2^n}_{x}{ \ket{xx}}}
\newcommand{\HON}{\ket{\psi_{\text{honest}}}}
\newcommand{\Lemma}{\paragraph{Lemma.}}
\newcommand{\PonB}{ \rho + \frac{5}{16}\delta\le \frac{3}{4} + \frac{1}{16} } 
\newcommand{\Cpa}{[n, \rho n, \delta n]}
%\setlength{\columnsep}{0.6cm}
\newcommand{\Jvv}{ \bar{J_{v}} } 
\newcommand{\Cvv}{ \tilde{C_{v}} } 

\newcommand{\Gz}{ G_{z}^{\delta} } 
\newcommand{ \Tann } {  \mathcal{T}\left( G, C_0 \right) }
\begin{document}



\newcommand{\dalg}[1]{\expp{#1 : \text{alg} \sim \tilde{\text{alg}}}}
\newcommand{\dsig}[1]{\expp{#1 : \sigma \sim \tilde{\sigma}}}
\newcommand{\calg}{c_{\text{alg}}}
\newcommand{\cbase}{c_{\text{base}}}



\title{Polytopes.} 
\author{David Ponarovsky}
\maketitle


%\abstract{We propose a new simple construction based on Tanner Codes, which yields a good LDPC testable code.} 

\begin{multicols*}{2}

  \section{Basics.} 
  \definition[Convex Polygon]{ $P$ will be said a convex polygon if for every $x,y \in P$ we have that any point $z$ that lays on the line between $x$ and $y$ belongs to $P$.} 
  

  \subsection{Different Constructions.} 
  Consider two different polytopes $P,Q \subset \mathbb{R}^{d}$ then we could construct a third polytope by: 
  \begin{enumerate}
    \item Intersection, taking the $P\cap Q \subset \mathbb{R}^{d}$
    \item Minkeoski sum, $P+Q = \left\{ p + q : p \in P, q \in Q \right\} \subset \mathbb{R}^{d}$
    \item Product, $P \times  Q = \left\{ \left( p,q \right):  p \in P, q \in Q \right\} \subset \mathbb{R}^{2d}$
  \end{enumerate}

  \paragraph{ $\mathcal{V}$ and $\mathcal{H}$ descriptors of polytopes.} Polytopes can be describe by both a convex hull or inequalities. There is theorem that state that any convex hull has a presentation defined by inequalities system.
  \lemma{A projection of an $\mathcal{H}$-polyhedron is also $\mathcal{H}$-polyhedron. }
\end{multicols*}
  \end{document}







\documentclass{article}

\input{usepackage.tex}
\input{newcommands.tex}

\title{PCP - Huji Course, Ex 1.} 
\author{David Ponarovsky}
%\abstract{We propose a new simple construction based on Tanner Codes, which yields a good LDPC testable code.} 

\newcommand{\FF}{\mathbb{F}}

\begin{document}
\maketitle
%\begin{multicols*}{2}
  
%  \section{Ex 1.}
%  \begin{claim}
%    Let $A$ be a random matrix in $M(\mathbb{F}_{2}^{k\times n})$ then for any non zero $x\in \FF$ we have that $Ax$ distributed uniformly.   
%  \end{claim}
%  \begin{proof}
%    By the fact that $x\neq 0$ there exists at least one coordinate $i \in [k]$ such that $x_{i}\neq 0$. Thus we have 
%
%    \begin{equation*}
%      \begin{split}    
%        \left( Ax \right)_{j} &= \sum_{k}{A_{jk}x_{k}} = \sum_{i \ neq k}{A_{jk}x_{k}}  + A_{ji}x_{i} \\ 
%        & =  \sum_{i \ neq k}{A_{jk}x_{k}}  + A_{ji}
%      \end{split}
%    \end{equation*}
%  Notice that due to the fact that $\FF_{2}$ is a field, there is exactly one assignment that satisfies the equation conditioned on all the values $A_{jk}$ where $j\neq k$.    
%
%  \begin{equation*}
%    \begin{split}
%      \prb{\left( Ax \right)_{j} = 1} &=  \sum_{ A_{jk}; k\neq i   }{\prb{\left( Ax \right)_{j} = 1 | A_{jk} ; k\neq i }\prb{ A_{jk} ; k\neq i }}  \\
%      &= \frac{1}{2} 
%    \end{split}
%  \end{equation*}
%  therefore any coordinate of $Ax$ distributed uniformly $\Rightarrow$ $Ax$ distributed uniformly. 
%  \end{proof}
%By the uniformity of $Ax$ we obtain that the expected Hamming wight of $Ax$ is : 
%\begin{equation*}
%  \begin{split}
%    \expp{|Ax| } = \expp{\sum_{i}^{n}{(Ax)_{i}}} = \frac{1}{2} n    
%  \end{split}
%\end{equation*} As the coordinates of $A_{x}$ are independent (each row of $A$ is sampled separately) we can use the Hoff' bound to conclude that: 
%\begin{equation*}
%  \begin{split}
%    \prb{| |Ax| - \expp{|Ax|}| \ge \left(\frac{1}{2} - \delta \right) n } & \le e^{-n \left(\frac{1}{2} - \delta \right)^{2} } 
%  \end{split}
%\end{equation*}
%Now we will use the union bound to show that any $x\in \FF_{2}^{k}$, $Ax$ is at weight at least $\delta$.  
%\begin{equation*}
%  \begin{split}
%    \prb{|Ax| \ge \delta : \forall x \in \FF_{2}^{k} } \ge 1 - |\FF_{2}^{k}| \cdot e^{-n \left(\frac{1}{2} - \delta \right)^{2} } 
%  \end{split}
%\end{equation*}
%Denote $k = \rho n$ and notice that the above holds when $\rho \ge \left(\frac{1}{2} - \delta \right)^{2} $  
%\section{Ex 2.}
%\begin{claim}
%  Let $v_{1},v_{2} .. v_{m}$ unit vectors in an inner-product space such that  $\braket{v_{i}, v_{j}} \le -2\varepsilon $ for all $i\neq j$, then $m \le \frac{1}{2\varepsilon}+1$.
%\end{claim}
%\begin{proof}
%  Let's us bound form both sides the norm of the summation $|\sum_{i}{v_{i}}|$. As the norm is by definition (construction) non-negative we are going to bound from the left by $0$, on the other hand we have that: 
%  \begin{equation*}
%    \begin{split}
%      0 \le |\sum_{v_{i}}{v_{i}}| = m + 2\sum_{i,j}{\braket{v_{i},v_{j}}} \le m - 2 \cdot \frac{m\left( m-1 \right)}{2} \cdot 2\varepsilon  
%    \end{split}
%  \end{equation*}
%  Thus we obtain $ m\left( 2(m-1)\varepsilon - 1  \right) \le 0 $ namely, $m \le \frac{1}{2\varepsilon} + 1$   
%\end{proof}
%Now, define the following product for $u,v \in \FF_{2}^{n} $, $\braket{v,u} = \sum_{i}{ (-1)^{v_{i}} (-1)^{\bar{u}_{i}}  }$ observes that: 
%\begin{enumerate}
%  \item $\braket{v,v} = \sum_{i}{ 1  } = n \ge 0 $.
%  \item $\braket{v,u} = \braket{u,v}$.
%  \item $\braket{ax + by,z} = (-1)^{a}\braket{x,z} + (-1)^{b}\braket{y,z}$. 
%\end{enumerate}
%
%Now the $v$'s corresponds to code with distance at least $d$ then, i.e for any codewords $v$ and $u$ disagree on at least $d$ coordinates, and therefore $\braket{v,u} \le$ \verb|agree|$-$ \verb|disagree| \verb|= n - 2 disagree| $=n -2d$. Now consider the normal codewords $\tilde{v_{1}} .. \tilde{v_{n}}$ and assume that 
%
%\begin{equation*}
%  \begin{split}
%    \braket{ \tilde{v_{i}},\tilde{v_{j}}} = \left(1 - 2\delta \right)= \frac{1}{n}\left( n -2 d(v_{i},v_{j}) \right) \le  \varepsilon
%  \end{split}
%\end{equation*}
%So if $d \ge \frac{1}{2} + \varepsilon$ we obtain the condition of the above claim. 
%
\section{Ex 1.}
Let $A$ be a random matrix in $M(\mathbb{F}_{2}^{k\times n})$. For any non-zero $x \in \FF$, we have that $Ax$ is distributed uniformly.
\begin{claim}
\end{claim}
\begin{proof}
By the fact that $x \neq 0$, there exists at least one coordinate $i \in [k]$ such that $x_i \neq 0$. Thus, we have
\begin{equation*}
    \begin{split}    
        \left( Ax \right)_{j} &= \sum_{k}{A_{jk}x_{k}} = \sum_{i \neq k}{A_{jk}x_{k}}  + A_{ji}x_{i} \\ 
        & =  \sum_{i \neq k}{A_{jk}x_{k}}  + A_{ji}
      \end{split}
\end{equation*}
Notice that due to the fact that $\FF_{2}$ is a field, there is exactly one assignment that satisfies the equation conditioned on all the values $A_{jk}$ where $j \neq k$.    

\begin{equation*}
    \begin{split}
      \prb{\left( Ax \right)_{j} = 1} &=  \sum_{ A_{jk}; k\neq i   }{\prb{\left( Ax \right)_{j} = 1 \mid A_{jk} ; k\neq i }\prb{ A_{jk} ; k\neq i }}  \\
      &= \frac{1}{2} 
    \end{split}
\end{equation*}
Therefore, any coordinate of $Ax$ is distributed uniformly $\Rightarrow$ $Ax$ is distributed uniformly. 
\end{proof}
By the uniformity of $Ax$, we obtain that the expected Hamming weight of $Ax$ is: 
\begin{equation*}
  \begin{split}
    \expp{|Ax| } = \expp{\sum_{i}^{n}{(Ax)_{i}}} = \frac{1}{2} n    
  \end{split}
\end{equation*} 
As the coordinates of $A_{x}$ are independent (each row of $A$ is sampled separately), we can use the Hoff's bound to conclude that: 
\begin{equation*}
  \begin{split}
    \prb{| |Ax| - \expp{|Ax|}| \ge \left(\frac{1}{2} - \delta \right) n } & \le e^{-n \left(\frac{1}{2} - \delta \right)^{2} } 
  \end{split}
\end{equation*}
Now, we will use the union bound to show that any $x \in \FF_{2}^{k}$, $Ax$ is of weight at least $\delta$.  
\begin{equation*}
  \begin{split}
    \prb{|Ax| \ge \delta : \forall x \in \FF_{2}^{k} } \ge 1 - |\FF_{2}^{k}| \cdot e^{-n \left(\frac{1}{2} - \delta \right)^{2} } 
  \end{split}
\end{equation*}
Denote $k = \rho n$ and notice that the above holds when $\rho \ge \left(\frac{1}{2} - \delta \right)^{2} $.  

\section{Ex 2.}
\begin{claim}
  Let $v_{1},v_{2}, \dots, v_{m}$ be unit vectors in an inner-product space such that  $\braket{v_{i}, v_{j}} \le -2\varepsilon $ for all $i\neq j$, then $m \le \frac{1}{2\varepsilon}+1$.
\end{claim}
\begin{proof}
  Let us bound the norm of the summation $|\sum_{i}{v_{i}}|$ from both sides. As the norm is non-negative by definition, we will bound it from the left by $0$. On the other hand, we have that: 
  \begin{equation*}
    \begin{split}
      0 \le |\sum_{v_{i}}{v_{i}}| = m + 2\sum_{i,j}{\braket{v_{i},v_{j}}} \le m - 2 \cdot \frac{m\left( m-1 \right)}{2} \cdot 2\varepsilon  
    \end{split}
  \end{equation*}
  Thus, we obtain $ m\left( 2(m-1)\varepsilon - 1  \right) \le 0 $, namely, $m \le \frac{1}{2\varepsilon} + 1$   
\end{proof}
Now, define the following product for $u,v \in \FF_{2}^{n} $, $\braket{v,u} = \sum_{i}{ (-1)^{v_{i}} (-1)^{\bar{u}_{i}}  }$ and observe that: 
\begin{enumerate}
  \item $\braket{v,v} = \sum_{i}{ 1  } = n \ge 0 $.
  \item $\braket{v,u} = \braket{u,v}$.
  \item $\braket{ax + by,z} = (-1)^{a}\braket{x,z} + (-1)^{b}\braket{y,z}$. 
\end{enumerate}

Now, if the $v$'s correspond to a code with distance at least $d$, then, for any codewords $v$ and $u$ that disagree on at least $d$ coordinates, we have that $\braket{v,u} \le$ \verb|agree|$-$ \verb|disagree| \verb|= n - 2 disagree| $=n -2d$. Now consider the normal codewords $\tilde{v_{1}} .. \tilde{v_{n}}$ and assume that 

\begin{equation*}
  \begin{split}
    \braket{ \tilde{v_{i}},\tilde{v_{j}}} = \left(1 - 2\delta \right)= \frac{1}{n}\left( n -2 d(v_{i},v_{j}) \right) \le  \varepsilon
  \end{split}
\end{equation*}
Therefore, if $d \ge \frac{1}{2} + \varepsilon$, we obtain the condition of the above claim.

\section{Ex 3.}
Consider the following process for decoding $a$,


\begin{algorithm}[H]
  \For {$ t \in [\tau]$}  {
    \For{$ i \in [n]$} {
      $x \sim_{u} \FF_{2}^{n}$ \\
      $a_{i}^{(t)} \leftarrow w\left( x \right) + w\left( \sigma_{i}(x) \right)$
    }
}
\For{$ i \in [n]$} {
  $\hat{a}_{i} \leftarrow [ \frac{1}{\tau}\sum_{t}^{\tau}{ a_{i}^{(t)}  }  ]$ 
}
\Return $ \hat{a}_{0}, \hat{a}_{1}, \hat{a}_{2} .. \hat{a}_{n} $
\end{algorithm}

\begin{claim}
  For $\tau = \Omega\left( \frac{1}{\varepsilon^{4}} \log\left( n \right) \right)$ The above decoding success to decode $w\left( x \right)$ with probability $\ge 1 - \frac{1}{n}$.
\end{claim}
\begin{proof}
  In this question we will say that $w$ agree on $x,\sigma_{i}(x)$ if both $x,\sigma_{i}(x)$ were either filliped or unflipped. Clearly if $w(x)$ agree with $w(\sigma_{i}(x))$ than  
  \begin{equation*}
    \begin{split}
      w\left( x \right) + w\left( \sigma_{i}(x) \right) & = H_{a}(x) + H_{a}(\sigma_{i}(x)) \\
      & = \sum_{i\neq j }{a_{j}(x_{j} + x_{j})} + a_{i}(x_{j} + 1 + x_{j})  = a_{j} \ \ ( \text{   neither of them were flipped. } ) \\
      & = 1 + H_{a}(x) + 1 + H_{a}(\sigma_{i}(x)) =  a_{i} \ \ (  \text{   both flipped.} )
    \end{split}
  \end{equation*}
  Thus we can bound the probability that $ w\left( x \right) + w\left( \sigma_{i}(x) \right) \neq a_{_i}$ by the probability that $w$ disagree on $x$ and that append at probability: 
  
  \begin{equation*}
    \begin{split}
     \xi := \left(1 - f(x)\right)f(\sigma_{i}(x)) + \left(1 - f(\sigma_{i}(x))\right)f(x)
    \end{split}
  \end{equation*}
  Now as we want bound $\xi$ we could think about the maximization problem under the restrictions that $f(x),f(\sigma_{i}(x) \le \frac{1}{2} - \varepsilon$. We know that the maximum lay on the boundary so we can assign $\frac{1}{2} - \varepsilon$ for each of the probabilities to obtain an upper bound. That will yield $\xi \le 2\cdot \left( \frac{1}{2} - \varepsilon \right)\left( \frac{1}{2} + \varepsilon \right)$, namely $\xi \le \frac{1}{2}- 2\varepsilon^{2}$. 
    Now the probability that a coordinate $i$ will rounded to the opposite side, that it $\hat{a}_{i} \neq a_{i}$ mean that arithmetic mean over $\tau$ experiments were $2\varepsilon^{2}$ far from the expectation. Which by Hoff' bound is bounded by: $e^{\tau 4\varepsilon^{4}}$. So using the union bound we obtain:     
    \begin{equation*}
      \begin{split}
        \prb{\text{ decoding success }} \ge 1 - n \cdot e^{\tau 4\varepsilon^{4}}
      \end{split}
    \end{equation*} 
    Therefore it's enough to take $\tau = O( \frac{1}{\varepsilon^{4}} \log(n)  )$ to obtain a decoder which run at time $O(  \frac{1}{\varepsilon^{4}} n \log(n))$ and success with heigh probability.   
\end{proof}

\section{Ex 3.}
Consider the following process for decoding $a$:

\begin{algorithm}[H]
  \For {$ t \in [\tau]$}  {
    \For{$ i \in [n]$} {
      $x \sim_{u} \FF_{2}^{n}$ \\
      $a_{i}^{(t)} \leftarrow w\left( x \right) + w\left( \sigma_{i}(x) \right)$
    }
}
\For{$ i \in [n]$} {
  $\hat{a}_{i} \leftarrow [ \frac{1}{\tau}\sum_{t}^{\tau}{ a_{i}^{(t)}  }  ]$ 
}
\Return $ \hat{a}_{0}, \hat{a}_{1}, \hat{a}_{2} .. \hat{a}_{n} $
\end{algorithm}

\begin{claim}
  For $\tau = \Omega\left( \frac{1}{\varepsilon^{4}} \log\left( n \right) \right)$ the above decoding succeeds in decoding $w\left( x \right)$ with probability $\ge 1 - \frac{1}{n}$.
\end{claim}
\begin{proof}
  In this question we will say that $w$ agrees on $x,\sigma_{i}(x)$ if both $x,\sigma_{i}(x)$ were either flipped or unflipped. Clearly, if $w(x)$ agrees with $w(\sigma_{i}(x))$ then  
  \begin{equation*}
    \begin{split}
      w\left( x \right) + w\left( \sigma_{i}(x) \right) & = H_{a}(x) + H_{a}(\sigma_{i}(x)) \\
      & = \sum_{i\neq j }{a_{j}(x_{j} + x_{j})} + a_{i}(x_{j} + 1 + x_{j})  = a_{j} \ \ ( \text{   neither of them were flipped. } ) \\
      & = 1 + H_{a}(x) + 1 + H_{a}(\sigma_{i}(x)) =  a_{i} \ \ (  \text{   both flipped.} )
    \end{split}
  \end{equation*}
  Thus, we can bound the probability that $ w\left( x \right) + w\left( \sigma_{i}(x) \right) \neq a_{_i}$ by the probability that $w$ disagrees on $x$ and that is at probability: 
  
  \begin{equation*}
    \begin{split}
     \xi := \left(1 - f(x)\right)f(\sigma_{i}(x)) + \left(1 - f(\sigma_{i}(x))\right)f(x)
    \end{split}
  \end{equation*}
  Now, to bound $\xi$, we can think about the maximization problem under the restrictions that $f(x),f(\sigma_{i}(x) \le \frac{1}{2} - \varepsilon$. We know that the maximum lies on the boundary, so we can assign $\frac{1}{2} - \varepsilon$ for each of the probabilities to obtain an upper bound. That will yield $\xi \le 2\cdot \left( \frac{1}{2} - \varepsilon \right)\left( \frac{1}{2} + \varepsilon \right)$, namely $\xi \le \frac{1}{2}- 2\varepsilon^{2}$. 
    Now, the probability that a coordinate $i$ will be rounded to the opposite side, i.e. $\hat{a}_{i} \neq a_{i}$, means that the arithmetic mean over $\tau$ experiments is $2\varepsilon^{2}$ far from the expectation. According to Hoff's bound, this is bounded by $e^{\tau 4\varepsilon^{4}}$. Thus, using the union bound, we obtain:     
    \begin{equation*}
      \begin{split}
        \prb{\text{ decoding success }} \ge 1 - n \cdot e^{\tau 4\varepsilon^{4}}
      \end{split}
    \end{equation*} 
    Therefore, it is enough to take $\tau = O( \frac{1}{\varepsilon^{4}} \log(n)  )$ to obtain a decoder that runs in time $O(  \frac{1}{\varepsilon^{4}} n \log(n))$ and succeeds with high probability.   
\end{proof}
\section{Ex 4.}
\subsection{(a)} We will prove that if for any $x$, $f$ interpolate well on $ x, x+1, .. ,x+d+1$ than any it interpolate well on every coordinates set at size $d+1$. Denote by $J \subset \FF_{q}$ at size $d+1$. Let's continue by induction on $\max J$. The base case $\max J = d+1 \Rightarrow J = \{1,2 .., d+1\}$ follow straightforwardly from the assumption. Assume the correctness for any $J$ such that $\max J \le x_{0}$ and consider $J^{\prime}$ such that $\max J^{\prime} = x_{0} + $. 
Now it given that $ S = \{ x_{0} - d, x_{0} - d + 1, .. x_{0} + 1 \}$ is well interpolating set,  so there exists coefficients $a_{1}, a_{d+1}$ such that $a_{d+1}f(x_{0}+1) = \sum_{x_{i}\in S/(x_{0}+1)}{a_{i}f\left( x_{i} \right)}$ On the overhand, for ant any $x_{i} \in S / (x_{0}+1)$ the union $K = x_{i} \cup J / (x_{0} + 1)$ is subset of $\FF_{q}$ at size $d+1$ such that $\max K \le x_{0}$.
Hence by induction assumption $K$ is well interpolating set and we can exchange any $f(x_{i})$ for $x_{i} \in S$ by a linear combination of $f(x_{i})$ for $x_{i} \in J/ ( x_{0} + 1)$. So in overall we obtain that $J$ is depended set, namely $f$ is well interpolate on $J$.  
\subsection{(b)} Define the function $g(x) = f(t^{-1}(x-s))$. Note that $q$ is prime, thus $\left( \FF_{q} / 0 , \cdot  \right)$ and  $ \left( \FF_{q} , +  \right)$ are groups and the inverse elements $-s,t^{-1}$ are exist and uniqs. Suppose that $y$ is a zero of $g \Rightarrow$ $f(t(y+s)) = g(y) = 0$, Hence the number of zeros of $f$ equals to the number of zeros of $g$, which means that their degree are equal $\Rightarrow$ $g$ is also a polynomial at degree at most $d$, $\Rightarrow a_{1}, a_{2} .. a_{d}$ are also the interpolation coefficients respecting to the interpolation set $\{ tx_{1}+s, tx_{2}+s, .., tx_{d} +s  \}$. 



\section{Ex (5).}
\subsection{(a)} As shown in the previous section, by the fact that $q$ is prime, we have that $g_{u,v}$ acts on $\FF_{q}^{m}$ by $g_{u,v}(x) = u + vx$, (for any $v \neq 0$). Thus, $f(g_{u,v}(x))$ is just a permutation over the values of $f$. As the number of zeros remains the same, we have that $f(g_{u,v}(x))$ is also a degree $d$ polynomial. Therefore, the restricted polynomial $f|_{L}$ corresponds to the restriction $L'$ of another polynomial obtained by taking $u' = 0$ and $v$ to be supported only on a single coordinate. Hence, the restricted polynomial can have at most $d$ zeros.
\subsection{(b)} As we have that $f$ is a polynomial of degree exactly $d$, there must be a monomial $x_{1}^{d_{1}}x_{2}^{d_{2}}..x_{k}^{d_{k}}$ such that $\sum_{i}{d_{i}} = d$. Denote by $g$ the sum of all those monomials, and by $v \in \FF_{q}^{n}$ a coordinate on which $g(v) \neq 0$ (if there is no such $v$, then we could write $f$ as a sum of monomials, each of degree at most $d-1$).

Now, as $g(v),t \neq 0$, we obtain that $g(vt)$ is equal exactly to $t^{d}\cdot c$ where $c$ is the sum of coefficients of each monomial of $g$ (also $c = g(1)$). As $f(x) - g(x)$ is a polynomial of degree $d-1$, it holds from the previous section that $(f-g)(vt)$ is also a polynomial of degree at most $d-1$. Thus, it cannot zero out $g(vt) \Rightarrow$ $f(vt) = (f - g)(vt) + g(vt)$ is also a polynomial of degree $d$.        

\section{Ex (6).}
\subsection{(a) and (b).} Let $F$ be a function from $ \{0,1.. d \}^{2} \rightarrow \FF_{q}$, we are going to define a $d$-degree polynomail $f : \FF_{q}^{2}\rightarrow \FF_{q}$  that agree with $F$ and show that is uniq. Notice that any polynomial could be written as $\sum_{i,j}{a_{i}a_{j}x^{i}y^{j}}$. Thus, the assignments of $(d+1)^{2}$ points define $(d+1)^{2}$ equations over $(d+1)^{2}$ variables. In addition, as the determinant of the matrix equals

\begin{figure}
\begin{equation*}
  \begin{split}
    \begin{bmatrix}
    1 & 0 & 0 &0 \cdot 0 &0^{2} \\
    1 & 1 & 0 &1 \cdot 0  &1^{2} \\
    1 & 1 & 0 &0 \cdot 1  &0^{2} \\
    1 & 1 & 1 &1 \cdot 1 &1^2 \\ 
    1 & 2 & 0 &2 \cdot 0  &2^2 \\ 
    1 & d & d & d\cdot d & d^{2}
    \end{bmatrix}
    \cdot 
    \begin{bmatrix}
      a_{00}  \\
      a_{01}  \\
      a_{10}  \\  
      a_{11}  \\
      a_{20}  \\
      a_{dd}  
    \end{bmatrix}
    = 
   \begin{bmatrix}
     F(0,0)  \\
     F(0,1)  \\
     F(1,0)  \\
     F(1,1)  \\
     F(2,0)  \\
     F(d,d)  
    \end{bmatrix}
  \end{split}
\end{equation*}
\caption{ Illustration of the equations system. The left system is a vadermonde matrix in which the $((x,y),(i,j))$ entry corresponds to $x^{i}y^{j}$ where $(x,y)$ are one of the points in $ (x,y)\in \{0..d \}^{2}$.}
\end{figure}

\begin{equation*}
  \begin{split}
    \sum_{\sigma \in S_{n}}{ \left( -1 \right)^{\sigma(\pi)}\prod_{i,j }{( x^{\sigma(i)_{1}}_{i}y^{\sigma(j)_{2}}_{j}  ) } } &= \prod_{i<j}{\left(x_{i} - x_{j}  \right)}\prod_{i<j}\left(y_{i} -y_{j} \right) \\
  &= \prod_{i<j}{\left(x_{i} - x_{j}  \right)}\prod_{i<j}\left(y_{i} -y_{j} \right) = \prod{i^{d-i}}\prod{j^{d-j}} \not{|} q
\end{split}
\end{equation*}
So the detriment is not zero, thus we can solve that system by gauss elimination and obtain unique solution. The solution is uniq and define the coefficients of $f$.  

\subsection{(c).}

Now consider a function $f : \FF_{q}^{2} \rightarrow \FF_{q}$ which any restriction of $f$ to any line is a polynomial at degree at most $d$. Let us denote by $f^{\prime}$ the polynomial defined by the point $ \{ f(0,0), f(0,1) .. f(d,d) \}$ namely, $f^{\prime}$ is the result of the interpolation over the restriction $f$ to $\{ 0,..,d \}^{2}$. Suppose that there is a point $(x_{0},y_{0})$ on which $f^{\prime},f$ disagree. Thus, the functions $g(x) := f(x,y_{0})$ and $g^{\prime}(x) := f^{\prime}\left( x,y_{0} \right)$ are not equal. But the points $\{ (1,y_{0}), (2,y_{0}) , .. , (d,y_{0})\}$ define a unique $d$-degree polynomial and therefore   


%\end{multicols*}
  \printbibliography 
\end{document}



\documentclass{article}

\usepackage[utf8]{inputenc}
\usepackage[a4paper, total={7.5in, 10in} ]{geometry}
\usepackage{braket}
\usepackage{xcolor}
\usepackage{amsmath}
\usepackage{amssymb}
\usepackage{amsfonts}
\usepackage{graphicx}
\usepackage{svg}
\usepackage{float}
\usepackage{tikz}
\usetikzlibrary{patterns, shapes.arrows}
\usepackage{adjustbox}
\usepackage{tikz-network}
\usepackage[ruled,lined,linesnumbered]{algorithm2e}
\usepackage{multicol}
\usepackage[backend=biber,style=alphabetic,sorting=ynt]{biblatex}
\usepackage{xcolor}
\usepackage{pgfplots}
\DeclareUnicodeCharacter{2212}{−}
\usepgfplotslibrary{groupplots,dateplot}
\pgfplotsset{compat=newest}

\addbibresource{sample.bib} %Import the bibliography file

\newcommand{\commentt}[1]{\textcolor{blue}{ \textbf{[COMMENT]} #1}}
\newcommand{\ctt}[1]{\commentt{#1}}
\newcommand{\prb}[1]{ \mathbf{Pr} \left[ {#1} \right]}
\newcommand{\expp}[1]{ \mathbf{E} \left[ {#1} \right]}
\newcommand{\onotation}[1]{\(\mathcal{O} \left( {#1}  \right) \)}
\newcommand{\ona}[1]{\onotation{#1}}
\newcommand{\PSI}{{\ket{\psi}}}
\newcommand{\LESn}{\ket{\psi_n}}
\newcommand{\LESa}{\ket{\phi_n}}
\newcommand{\LESs}{\frac{1}{\sqrt{n}}\sum_{i}{\ket{\left(0^{i}10^{n-i}\right)^{n}}}}
\newcommand{\Hn}{\mathcal{H}_{n}}
\newcommand{\Ep}{\frac{1}{\sqrt{2^n}}\sum^{2^n}_{x}{ \ket{xx}}}
\newcommand{\HON}{\ket{\psi_{\text{honest}}}}
\newcommand{\Lemma}{\paragraph{Lemma.}}
\newcommand{\PonB}{ \rho + \frac{5}{16}\delta\le \frac{3}{4} + \frac{1}{16} } 
\newcommand{\Cpa}{[n, \rho n, \delta n]}
%\setlength{\columnsep}{0.6cm}
\newcommand{\Jvv}{ \bar{J_{v}} } 
\newcommand{\Cvv}{ \tilde{C_{v}} } 

\newcommand{\Gz}{ G_{z}^{\delta} } 
\newcommand{ \Tann } {  \mathcal{T}\left( G, C_0 \right) }
\begin{document}


\title{Simple LTC Good LDPC Codes} 
\author{David Ponarovsky}
\maketitle
\abstract{We propose a new simple construction based on Tanner Codes, which yields a good LDPC testable code.} 
\begin{multicols*}{2}
  \paragraph{ex1.} Find a simple description of the work-function algorithm in the case of uniform metric space. 
  \paragraph{ex2.} Consider the following $3$-point metric space, $w\left( a,b \right) = 1 $ and $w\left( \cdot, c  \right) = M $. The initial configuration is $ \left\{ b,c \right\}$ ($2$ servers). Show that randomized competitive ratio, for some value of $M$ is $ > H_{2} = 1 + \frac{1}{2}$. 
  \paragraph{ex3.} Show that randomized marking algortihm cannot be $c$-competitive against the adaptive online adversary, for $c=o\left( k \right)$. 
  \paragraph{ex4 - Ski Rental.} At each step, the adversary decides etiher continue or stop. Stop terminate the game. If it continues, the online algorithm decides, etiher rent or buy. Rent costs $1$ Buy costs $M > 1$. Deisgn a primal-dual randomized online ski-rental algorithm with better than $2$ competitive ratio.  
  \paragraph{ex5.} Prove Yao's minimax principle. 
  
  \newcommand{\dalg}[1]{\expp{#1 : \text{alg} \sim \tilde{\text{alg}}}}
  \newcommand{\dsig}[1]{\expp{#1 : \sigma \sim \tilde{\sigma}}}
  \newcommand{\calg}{c_{\text{alg}}}
  \newcommand{\cbase}{c_{\text{base}}}
  \begin{equation*}
    \begin{split}
      & \forall \text{rand. } \tilde{\text{alg}} \  \exists  \ \sigma  \\
      & \ \ \dalg{ \calg\left( \sigma \right) } \ge c \cdot   \cbase\left( \sigma \right) 
      & \leftrightarrow \exists \ \text{rand.} \ \tilde{\sigma} \forall \text{alg} \\  
      & \ \ \dsig{  \calg\left( \sigma \right)  } \ge c \dsig{ \cbase \left( \sigma \right)  } 
      \end{split}
  \end{equation*}
\end{multicols*}
\printbibliography 
\end{document}






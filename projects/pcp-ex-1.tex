\documentclass{article}

\input{usepackage.tex}
\input{newcommands.tex}

\title{PCP - Huji Course, Ex 1.} 
\author{David Ponarovsky}
%\abstract{We propose a new simple construction based on Tanner Codes, which yields a good LDPC testable code.} 

\newcommand{\FF}{\mathbb{F}}

\begin{document}
%\begin{multicols*}{2}
  
  \section{Ex 1.}
  \begin{claim}
    Let $A$ be a random matrix in $M(\mathbb{F}_{2}^{k\times n})$ then for any non zero $x\in \FF$ we have that $Ax$ distributed uniformly.   
  \end{claim}
  \begin{proof}
    By the fact that $x\neq 0$ there exists at least one coordinate $i \in [k]$ such that $x_{i}\neq 0$. Thus we have 

    \begin{equation*}
      \begin{split}    
        \left( Ax \right)_{j} &= \sum_{k}{A_{jk}x_{k}} = \sum_{i \ neq k}{A_{jk}x_{k}}  + A_{ji}x_{i} \\ 
        & =  \sum_{i \ neq k}{A_{jk}x_{k}}  + A_{ji}
      \end{split}
    \end{equation*}
  Notice that due to the fact that $\FF_{2}$ is a field, there is exactly one assignment that satisfies the equation conditioned on all the values $A_{jk}$ where $j\neq k$.    

  \begin{equation*}
    \begin{split}
      \prb{\left( Ax \right)_{j} = 1} &=  \sum_{ A_{jk}; k\neq i   }{\prb{\left( Ax \right)_{j} = 1 | A_{jk} ; k\neq i }\prb{ A_{jk} ; k\neq i }}  \\
      &= \frac{1}{2} 
    \end{split}
  \end{equation*}
  therefore any coordinate of $Ax$ distributed uniformly $\Rightarrow$ $Ax$ distributed uniformly. 
  \end{proof}
By the uniformity of $Ax$ we obtain that the expected Hamming wight of $Ax$ is : 
\begin{equation*}
  \begin{split}
    \expp{|Ax| } = \expp{\sum_{i}^{n}{(Ax)_{i}}} = \frac{1}{2} n    
  \end{split}
\end{equation*} As the coordinates of $A_{x}$ are independent (each row of $A$ is sampled separately) we can use the Hoff' bound to conclude that: 
\begin{equation*}
  \begin{split}
    \prb{| |Ax| - \expp{|Ax|}| \ge \left(\frac{1}{2} - \delta \right) n } & \le e^{-n \left(\frac{1}{2} - \delta \right)^{2} } 
  \end{split}
\end{equation*}
Now we will use the union bound to show that any $x\in \FF_{2}^{k}$, $Ax$ is at weight at least $\delta$.  
\begin{equation*}
  \begin{split}
    \prb{|Ax| \ge \delta : \forall x \in \FF_{2}^{k} } \ge 1 - |\FF_{2}^{k}| \cdot e^{-n \left(\frac{1}{2} - \delta \right)^{2} } 
  \end{split}
\end{equation*}
Denote $k = \rho n$ and notice that the above holds when $\rho \ge \left(\frac{1}{2} - \delta \right)^{2} $  
\section{Ex 2.}
\begin{claim}
  Let $v_{1},v_{2} .. v_{m}$ unit vectors in an inner-product space such that  $\braket{v_{i}, v_{j}} \le -2\varepsilon $ for all $i\neq j$, then $m \le \frac{1}{2\varepsilon}+1$.
\end{claim}
\begin{proof}
  Let's us bound form both sides the norm of the summation $|\sum_{i}{v_{i}}|$. As the norm is by definition (construction) non-negative we are going to bound from the left by $0$, on the other hand we have that: 
  \begin{equation*}
    \begin{split}
      0 \le |\sum_{v_{i}}{v_{i}}| = m + 2\sum_{i,j}{\braket{v_{i},v_{j}}} \le m - 2 \cdot \frac{m\left( m-1 \right)}{2} \cdot 2\varepsilon  
    \end{split}
  \end{equation*}
  Thus we obtain $ m\left( 2(m-1)\varepsilon - 1  \right) \le 0 $ namely, $m \le \frac{1}{2\varepsilon} + 1$   
\end{proof}
Now, define the following product for $u,v \in \FF_{2}^{n} $, $\braket{v,u} = \sum_{i}{ (-1)^{v_{i}} (-1)^{\bar{u}_{i}}  }$ observes that: 
\begin{enumerate}
  \item $\braket{v,v} = \sum_{i}{ 1  } = n \ge 0 $.
  \item $\braket{v,u} = \braket{u,v}$.
  \item $\braket{ax + by,z} = (-1)^{a}\braket{x,z} + (-1)^{b}\braket{y,z}$. 
\end{enumerate}

Now the $v$'s corresponds to code with distance at least $d$ then, i.e for any codewords $v$ and $u$ disagree on at least $d$ coordinates, and therefore $\braket{v,u} \le$ \verb|agree|$-$ \verb|disagree| \verb|= n - 2 disagree| $=n -2d$. Now consider the normal codewords $\tilde{v_{1}} .. \tilde{v_{n}}$ and assume that 

\begin{equation*}
  \begin{split}
    \braket{ \tilde{v_{i}},\tilde{v_{j}}} = \left(1 - 2\delta \right)= \frac{1}{n}\left( n -2 d(v_{i},v_{j}) \right) \le  \varepsilon
  \end{split}
\end{equation*}
So if $d \ge \frac{1}{2} + \varepsilon$ we obtain the condition of the above claim. 

\section{Ex 3.}
Consider the following process for decoding $a$,


\begin{algorithm}[H]
  \For {$ t \in [\tau]$}  {
    \For{$ i \in [n]$} {
      $x \sim_{u} \FF_{2}^{n}$ \\
      $a_{i}^{(t)} \leftarrow w\left( x \right) + w\left( \sigma_{i}(x) \right)$
    }
}
\For{$ i \in [n]$} {
  $\hat{a}_{i} \leftarrow [ \frac{1}{\tau}\sum_{t}^{\tau}{ a_{i}^{(t)}  }  ]$ 
}
\Return $ \hat{a}_{0}, \hat{a}_{1}, \hat{a}_{2} .. \hat{a}_{n} $
\end{algorithm}

\begin{claim}
  For $\tau = \Omega\left( \frac{1}{\varepsilon^{4}} \log\left( n \right) \right)$ The above decoding success to decode $w\left( x \right)$ with probability $\ge 1 - \frac{1}{n}$.
\end{claim}
\begin{proof}
  In this question we will say that $w$ agree on $x,\sigma_{i}(x)$ if both $x,\sigma_{i}(x)$ were either filliped or unflipped. Clearly if $w(x)$ agree with $w(\sigma_{i}(x))$ than  
  \begin{equation*}
    \begin{split}
      w\left( x \right) + w\left( \sigma_{i}(x) \right) & = H_{a}(x) + H_{a}(\sigma_{i}(x)) \\
      & = \sum_{i\neq j }{a_{j}(x_{j} + x_{j})} + a_{i}(x_{j} + 1 + x_{j})  = a_{j} \ \ ( \text{   neither of them were flipped. } ) \\
      & = 1 + H_{a}(x) + 1 + H_{a}(\sigma_{i}(x)) =  a_{i} \ \ (  \text{   both flipped.} )
    \end{split}
  \end{equation*}
  Thus we can bound the probability that $ w\left( x \right) + w\left( \sigma_{i}(x) \right) \neq a_{_i}$ by the probability that $w$ disagree on $x$ and that append at probability: 
  
  \begin{equation*}
    \begin{split}
     \xi := \left(1 - f(x)\right)f(\sigma_{i}(x)) + \left(1 - f(\sigma_{i}(x))\right)f(x)
    \end{split}
  \end{equation*}
  Now as we want bound $\xi$ we could think about the maximization problem under the restrictions that $f(x),f(\sigma_{i}(x) \le \frac{1}{2} - \varepsilon$. We know that the maximum lay on the boundary so we can assign $\frac{1}{2} - \varepsilon$ for each of the probabilities to obtain an upper bound. That will yield $\xi \le 2\cdot \left( \frac{1}{2} - \varepsilon \right)\left( \frac{1}{2} + \varepsilon \right)$, namely $\xi \le \frac{1}{2}- 2\varepsilon^{2}$. 
    Now the probability that a coordinate $i$ will rounded to the opposite side, that it $\hat{a}_{i} \neq a_{i}$ mean that arithmetic mean over $\tau$ experiments were $2\varepsilon^{2}$ far from the expectation. Which by Hoff' bound is bounded by: $e^{\tau 4\varepsilon^{4}}$. So using the union bound we obtain:     
    \begin{equation*}
      \begin{split}
        \prb{\text{ decoding success }} \ge 1 - n \cdot e^{\tau 4\varepsilon^{4}}
      \end{split}
    \end{equation*} 
    Therefore it's enough to take $\tau = O( \frac{1}{\varepsilon^{4}} \log(n)  )$ to obtain a decoder which run at time $O(  \frac{1}{\varepsilon^{4}} n \log(n))$ and success with heigh probability.   
\end{proof}

\section{Ex 4.}


%\end{multicols*}
  \printbibliography 
\end{document}



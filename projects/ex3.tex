\documentclass{article}

\usepackage[utf8]{inputenc}
\usepackage[a4paper, total={7.5in, 10in} ]{geometry}
\usepackage{braket}
\usepackage{xcolor}
\usepackage{amsmath}
\usepackage{amssymb}
\usepackage{amsfonts}
\usepackage{graphicx}
\usepackage{svg}
\usepackage{float}
\usepackage{tikz}
\usetikzlibrary{patterns, shapes.arrows}
\usepackage{adjustbox}
\usepackage{tikz-network}
\usepackage[linesnumbered]{algorithm2e}
\usepackage{multicol}
\usepackage[backend=biber,style=alphabetic,sorting=ynt]{biblatex}
\usepackage{xcolor}
\usepackage{pgfplots}
\DeclareUnicodeCharacter{2212}{−}
\usepgfplotslibrary{groupplots,dateplot}
\pgfplotsset{compat=newest}

\addbibresource{sample.bib} %Import the bibliography file

\newcommand{\commentt}[1]{\textcolor{blue}{ \textbf{[COMMENT]} #1}}
\newcommand{\ctt}[1]{\commentt{#1}}
\newcommand{\prb}[1]{ \mathbf{Pr} \left[ {#1} \right]}
\newcommand{\expp}[1]{ \mathbf{E} \left[ {#1} \right]}
\newcommand{\onotation}[1]{\(\mathcal{O} \left( {#1}  \right) \)}
\newcommand{\ona}[1]{\onotation{#1}}
\newcommand{\PSI}{{\ket{\psi}}}
\newcommand{\LESn}{\ket{\psi_n}}
\newcommand{\LESa}{\ket{\phi_n}}
\newcommand{\LESs}{\frac{1}{\sqrt{n}}\sum_{i}{\ket{\left(0^{i}10^{n-i}\right)^{n}}}}
\newcommand{\Hn}{\mathcal{H}_{n}}
\newcommand{\Ep}{\frac{1}{\sqrt{2^n}}\sum^{2^n}_{x}{ \ket{xx}}}
\newcommand{\HON}{\ket{\psi_{\text{honest}}}}
\newcommand{\Lemma}{\paragraph{Lemma.}}
\newcommand{\PonB}{ \rho + \frac{5}{16}\delta\le \frac{3}{4} + \frac{1}{16} } 
\newcommand{\Cpa}{[n, \rho n, \delta n]}
%\setlength{\columnsep}{0.6cm}
\newcommand{\Jvv}{ \bar{J_{v}} } 
\newcommand{\Cvv}{ \tilde{C_{v}} } 

\newcommand{\Gz}{ G_{z}^{\delta} } 
\newcommand{ \Tann } {  \mathcal{T}\left( G, C_0 \right) }
\begin{document}



\newcommand{\dalg}[1]{\expp{#1 : \text{alg} \sim \tilde{\text{alg}}}}
\newcommand{\dsig}[1]{\expp{#1 : \sigma \sim \tilde{\sigma}}}
\newcommand{\calg}{c_{\text{alg}}}
\newcommand{\cbase}{c_{\text{base}}}


\title{Online Computation, Ex 3.} 
\author{David Ponarovsky}
\maketitle
%\abstract{We propose a new simple construction based on Tanner Codes, which yields a good LDPC testable code.} 

\begin{multicols*}{2}
  \paragraph{ex1.} Consider the experts setting with gains: $g_{i,t} \in \left[ 0,1 \right]$ is the gain of expert $i$ at step $t$. Hedge updates:  
  \begin{equation*}
    \begin{split}
      P_{i,t+1}= \frac{e^{\eta G_{i,t}}}{\sum_{j}{ e^{\eta G_{j,t}}} }
    \end{split}
  \end{equation*} where $G_{i,t} = \sum_{s\le t}{g_{i,t}} $. Prove that the regret of Hedge at time $T$ is $O\left( \sqrt{T\log n} \right)$, for a good choice of the learning rate $\eta$, against the adaptive adversary.  
  \paragraph{Solution.} Define the potential $\psi\left( t \right) =  \sum_{j}{ e^{\eta G_{i,t}}}$ and notice that by the fact that $e^{x}$ is positive function we have that $\psi\left( t \right) \ge e^{\eta \max_j G_{i,j}}$. In addition: 
  \begin{equation*}
    \begin{split}
      & \psi\left( t + 1 \right) =  \sum_{j}{ e^{\eta G_{j,t+1}}} =\sum_{j}{ e^{\eta G_{j,t} + \eta g_{j,t+1}}}  \\
      %& \Rightarrow   \psi\left( t \right) \le \psi\left( t + 1 \right)\le \psi\left( t \right)e^{\eta} \\ 
      & \le \sum_{j}{ e^{\eta G_{j,t}  }\left(  1 + c_{1} \eta g_{j,t+1} \right) }= \psi\left( t \right)  +  c_{1}\eta \psi\left( t \right) \expp{g_{t+1}}  \\
      & \le = e^{\mu} \psi\left( t \right) \left( 1 +  c_{2} \right) \psi\left( t \right) \expp{g_{t+1}}  \\
      & \le \prod_{t}{ \left( 1 + c_{1} \expp{g_{t}}\right) }\left( \psi\left( 0 \right) \right) \le e ^{c_{2} \expp{ \sum{g_{t} } }}\psi\left(0  \right) \\ 
      & \le e^{c_{2}\expp{G_{t}}} \cdot e^{\eta}n 
    \end{split}
  \end{equation*}
  So after $T$ steps, by taking the logaritm of both sides, we obtain that the regret is bounded by $ R_{T} \le   $.
  \paragraph{ex2.} Show a lower bound of $\Omega\left( \sqrt{T} \right)$ in the experts setting on the regret of any online algorithm against the oblivious adversary. 

  \paragraph{Solution.} solution. 
  \paragraph{ex3.}Consider a system of linear inequalities $Ax \ge b$, where $A\in [0, \infty]^{m\times n} , b \in  [0, \infty]^{m}$, and unknown $x \in  [0, \infty]^{ n}$. (we are seeking a non-negative solution). An $\varepsilon$-approximate solution $x\ge 0$ satisfies $Ax\ge b - \varepsilon \mathbf{1}$. Suppose we have an efficient procedure for following problem: Given $p \in [0,1]^{m}, \sum_{i \in [m]}{p_{i}} = 1$, decide if exists $x \ge 0, p^{\top}Ax \ge p^{\top}b$. Show how to find an $\varepsilon$-approximate solution to $Ax \ge b$. Analyze the run-time. 
  \paragraph{Solution.} solution. 
  \paragraph{ex4.} Recall that we showed, for $EXP$ updates, that w.p $1 - \delta$ 
  \begin{equation*}
    \begin{split}
      &  R{T} \le \beta n T + \gamma T + \left( 1 + \beta \right)\eta + \frac{\ln\left( \delta^{-1} n \right) }{\beta} + \frac{\ln n}{\eta}  
    \end{split}
  \end{equation*}
  Infer that for the right choice of $\beta,\gamma, \eta$ 
  \begin{equation*}
    \begin{split}
      \expp{R_{T}} = O\left( \sqrt{T n \ln n } \right)
    \end{split}
  \end{equation*}
\end{multicols*}
  \printbibliography 
\end{document}






